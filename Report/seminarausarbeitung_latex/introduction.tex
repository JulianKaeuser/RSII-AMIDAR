\chapter{Introduction}
\label{cha:introduction}

In this lab, our task was to find an optimized combination of mathematical modification, code and hardware configuration
of a GPS signal acquisition algorithm for the RS AMIDAR processor. The quality of the solution is measured in two metrics:
required ticks for the computation in the AMIDAR simulato, further called \emph{performance optimization} and consumed energy
as measured by the simulator. For both goals, dedicated solutions in all aspects are allowed, but not required. All code must be
written in Java 1.4 in order to obatin the compatibility with the current AMIDAR toolchain.

Our presented solutions include optimizations in the following fields:
\begin{enumerate}
    \item \textbf{Mathematical Conversions} \\ 
    The algorithm represents a cross convolution between a vector of LA-1 codes and the samples of the GPS receiver. Any mathematical 
    conversion which yields the same results can be applied. The ability to map the applied conversions efficiently in code and onto the hardware
    may be an aspect to watch while applying these conversions.\\
    \item \textbf{Code Style Optimizations}\\
    Although the required programming language is only the Java 1.4 standard, many different ways to implement the mathematical description
    are possible. Additionally, the effects on the hardware mapping, orders of execution, parallelism and (eventually missing) compiler optimizations
    have to be taken into account.\\
    \item \textbf{Hardware Configuration}\\
    The AMIDAR processor is parametrizable in some aspects and features a Coarse-Grained Reconfigurable Array (CGRA), which allows a high speedup if used and configured
    properly. This configuration, called \emph{composition}, can particularly speedup the execution or affect the overall energy consumption.
\end{enumerate}

In this report, an overview of our solutions in the presented fields is given in the particular order. The report closes with an evaluation of the results.

\section{Original Algorithm}
\label{sec:originalAlgorithm}

The subject of all optimizations is the following algorithm, which presents the extraction of a peak in the cross correlation between a sampled signal and a vector of 
LA-1 codes. The cross correlation is presented as the circular convolution of the inversed code vector with the sample vector, which can be converted to a DFT/IDFT pair:

eqns here